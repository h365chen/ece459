\input{configuration}

\title{Lecture 1 --- Programming for Performance}

\author{Jeff Zarnett and Patrick Lam \\ \small \texttt{jzarnett@uwaterloo.ca, patrick.lam@uwaterloo.ca}}
\institute{Department of Electrical and Computer Engineering \\
  University of Waterloo}
\date{\today}


\begin{document}

\begin{frame}
  \titlepage

 \end{frame}

\begin{frame}
\frametitle{Lucas, SE2015}

\begin{center}
\includegraphics[width=.9\textwidth]{images/lucas_waterton.jpg}
\end{center}
\end{frame}

\begin{frame}
\frametitle{The Continental Divide Trail}
\begin{center}
\includegraphics[width=.3\textwidth]{images/cdt.png}
~\\
3000 miles, 5 months.\\
How to do faster?
\end{center}
\end{frame}

\begin{frame}
\frametitle{Course Syllabus}

As our first order of business, let's go over the course syllabus.

\end{frame}

\begin{frame}
\frametitle{Collaborative Course}

The source material for the ECE~459 notes \& slides is open-sourced via Github. 

If you find an error in the notes/slides, or have an improvement, go to \url{https://github.com/jzarnett/ece459} and open an issue. 

You can submit a pull request (changes) for me to look at and incorporate!

\end{frame}



\begin{frame}
\frametitle{Performance!}

I'm certain you know what ``programming'' means. But define ``performance''. 

Making a program ``fast''. 

Alright, but what does it mean for a program to be fast?


\end{frame}

\begin{frame}
\frametitle{What is Fast?}

\begin{center}
\includegraphics[width=0.8\textwidth]{images/blackbird.jpg}
\end{center}
\hfill SR-71B Blackbird. Photo Credit: USAF / Judson Brohmer
\end{frame}

\begin{frame}
\frametitle{What is Fast?}

Program execution as completion of some number of items---things to do. 

We have two concepts:\\
\quad (1) items per unit time (bandwidth---more is better)\\
\quad (2) time per item (latency---less is better). 

Improving on either of these will make your program ``faster'' in some sense.

\end{frame}

\begin{frame}
\frametitle{Our Metrics}

In a way they are somewhat related. 

If we reduce the time per item from 5~s to 4~s, it means an increase of 12 items per minute to 15 items per minute.

...if the conditions are right. 

Hopefully we could improve both metrics; sometimes we'll have to pick one.



\end{frame}

\begin{frame}
\frametitle{Bandwidth}

This measures how much work can get 
done simultaneously.

Parallelization improves the number
of items per unit time. 

``Never underestimate the bandwidth of a station wagon full of tapes hurtling down the highway.''

\hfill --- Andrew S. Tanenbaum

\end{frame}




\begin{frame}
\frametitle{Latency}

This measures how much time it takes to do
any one particular task.

Also called response time.

It doesn't tend to get measured as often as bandwidth, but it's especially
important for tasks where people are involved. 

Google cares, which is why they provide the {\tt 8.8.8.8} DNS servers.



\end{frame}


\begin{frame}
\frametitle{Sadako and the Hundred Paper Airplanes}

 Say you need to make
100 paper airplanes. What's the fastest way of doing this?

\begin{center}
\includegraphics[width=0.4\textwidth]{images/paper-airplane.png}
\end{center}

\end{frame}


\begin{frame}
\frametitle{Bandwidth vs Latency}


\begin{center}
\begin{tikzpicture}
  \draw (-4,0) -- (4,0) node[below right,xshift=-4em] {high latency};
  \draw (0,-1.5) -- (0,1.5) node[right] {high bandwidth};
\end{tikzpicture}
\end{center}

We will focus on completing the items, not on transmitting information.

The above example makes the difference between bandwidth and latency clear.

\end{frame}


\begin{frame}
\frametitle{Improving Latency}

A good way of writing faster code is by improving single-threaded 
performance. 

There is a limit to how much you can
improve single-threaded performance.

Any improvements here
may also help with the parallelized version. 

On the other hand, faster
sequential algorithms may not parallelize as well. 


\end{frame}




\begin{frame}
\frametitle{Profiling}

\begin{center}
	\includegraphics[width=0.3\textwidth]{images/clues.png}
\end{center}

You can't successfully make your code 
faster if you don't know why it's slow. 

Intuition seems to often be
wrong here. 

Run your program with realistic workloads under a profiling tool.

``Don't guess; measure''.

\end{frame}


\begin{frame}
\frametitle{Exercise Time!}

Let's take a quick minute to visit \url{http://computers-are-fast.github.io/} 

\end{frame}

\begin{frame}
\frametitle{Self Assessment}

Are the results surprising to you? 

Did you do really well or really badly? 

Chances are that you got some right and some wrong... and the ones that were wrong were not just a little wrong, but off by several orders of magnitude. 

Moral of the story is: don't just guess at what the slow parts of your code are. 

It's okay to have a theory as a starting point, but test your theory.

\end{frame}

\begin{frame}
\frametitle{Do Less Work}

A surefire way to be faster is to omit unnecessary
work. 

Two (related) ways of omitting work are:\\
\quad (1) avoid calculating intermediate results that you don't actually need;\\
\quad (2) compute results to only the accuracy that you need in the final output.


\end{frame}

\begin{frame}
\frametitle{A Note About Logging}

\begin{center}
	\includegraphics[width=0.5\textwidth]{images/logging.jpeg}
\end{center}

Producing text output to a log file or to a console screen is surprisingly expensive for the computer. 

\end{frame}

\begin{frame}
\frametitle{Caching}

A hybrid between ``do less work'' and ``be smarter'' is caching. 

Store the results of expensive, side-effect-free, operation
(potentially I/O and computation) and reuse them as long as you
know that they are still valid. 

Caching is really important in certain situations.


\end{frame}



\begin{frame}
\frametitle{Be Prepared}

\begin{center}
	\includegraphics[width=0.5\textwidth]{images/beprepared.jpg}
\end{center}

\end{frame}


\begin{frame}
\frametitle{Be Prepared}

If you know something that the user is going to ask for in advance, you can have it at the ready to provide upon request. 

Example: users want an Excel export of statistics on customs declarations. 

Report generation takes a while, and it means a long wait. 

Alternative: data pre-generated and stored in database (updated as necessary).

Then putting it into Excel is simple and the report is available quickly.


\end{frame}




\begin{frame}
\frametitle{Be Smart}

\begin{center}
	\includegraphics[width=0.4\textwidth]{images/besmart.jpg}
\end{center}


An improved algorithm includes better asymptotic performance
as well as smarter data structures and smaller constant factors.


Compiler optimizations (which we'll discuss in this course) help with
getting smaller constant factors. 

We may also need to be aware of cache
and data locality/density issues.

\end{frame}



\begin{frame}
\frametitle{Checking out from the Library}

Sometimes you can find this type of improvements in your choice of
libraries.

Use a more specialized library which does the
task you need more quickly.
 
It's a hard decision sometimes. 

Libraries may be better and more reliable than the code you can write yourself. 

Or it might be better to write your own implementation that is optimized especially for your use case.

\end{frame}


\begin{frame}
\frametitle{My Money Bin!}

\begin{center}
	\includegraphics[width=0.5\textwidth]{images/scrooge.jpg}
\end{center}

\end{frame}


\begin{frame}
\frametitle{Throw Money at the Problem}

Once upon a time, it was okay to write code with terrible performance on the theory that next year's CPUs would make it run acceptably. 

Spending a ton of time optimizing your code to run on today's processors was a waste of time. 

Well, those days seem to be over; CPUs are not getting much faster these days (evolutionary rather than revolutionary change). 

\end{frame}



\begin{frame}
\frametitle{Spend Money to Save Money}

What if the CPU is not the limiting factor: your code might be I/O-bound.\\
\quad Buy some SSDs! 

You might be swapping out to disk, which kills performance .\\
\quad Add RAM. 

Profiling is key here, to find out what the slow parts of execution are. 

Spending a few thousand dollars on better hardware is often much cheaper than paying programmers to spend their time to optimize the code.


\end{frame}


\begin{frame}
\frametitle{I'll Write Assembly!}

\begin{center}
	\includegraphics[width=0.5\textwidth]{images/miniature-paintbrush.jpg}
\end{center}

What about outsmarting the compiler and writing assembly by hand? 

\end{frame}



\begin{frame}
\frametitle{MOV R1,R3}

Compilers are going to be better at generating assembly than you are. 

Furthermore, CPUs may accept the commands in x86 assembly (or whatever your platform is) but internally they don't operate on those commands directly. 

They rearrange and reinterpret and do their own thing. 


\end{frame}



\begin{frame}
\frametitle{Anecdote Time}

``The report generation has been running for three hours; I think it's stuck.''

Nope, it reached a 30 minute time limit and got killed. 

How do I speed up this task to get it under the 30 minute time limit?

\end{frame}



\begin{frame}
\frametitle{Fly, My Pretties!}

We can do more things at a time.

Often, it is easier to just throw more
resources at the problem: use a bunch of CPUs at the same time.

 We
will study how to effectively throw more resources at problems.

In general, parallelism improves bandwidth, but not latency.


\end{frame}



\begin{frame}
\frametitle{Kinds of Parallelism}

Different problems are amenable to different sorts of parallelization. 

For instance, in a web server, we
can easily parallelize simultaneous requests. 

On the other hand, it's hard
to parallelize a linked list traversal. (Why?)



\end{frame}



\begin{frame}
\frametitle{Pipelining}

A key concept is pipelining. 

All modern CPUs do this,
but you can do it in your code too. 

Think of an assembly line: you can split
a task into a set of subtasks and execute these subtasks in parallel.


\end{frame}




\begin{frame}
\frametitle{Hardware}

To get parallelism, we need to have multiple instruction
streams executing simultaneously. 

We can do this by increasing the
number of CPUs: we can use multicore processors, SMP (symmetric
multiprocessor) systems, or a cluster of machines.

We get different
communication latencies with each of these choices.

We can also use more exotic hardware, like graphics processing units
(GPUs).

\end{frame}



\begin{frame}
\frametitle{Mo' Parallelism, Mo' Problems}
You may have noticed that it is easier to do a project when it's just
you rather than being you and a team. 

\begin{center}
	\includegraphics[width=0.5\textwidth]{images/group-projects.jpg}
\end{center}

The same applies to code.

Here are some of the issues with parallel code.


\end{frame}



\begin{frame}
\frametitle{Exceptions}

Some domains are ``embarassingly parallel''; these problems
don't apply. 

It's easy to communicate the problem to all of the processors and to get the answer back.
 
The processors don't need to talk to each other to compute. 

The canonical
example is Monte Carlo integration.


\end{frame}



\begin{frame}
\frametitle{Limitations}

First, a task can't start processing until it knows what it
is supposed to process. 

Coordination overhead is an issue. 

If the
problem lacks a succinct description, parallelization can be
difficult. 

Also, the task needs to combine its result with the other
tasks.

\end{frame}



\begin{frame}
\frametitle{Inherently Sequential Problems}

``Inherently sequential'' problems are an issue. 

In a sequential 
program, it's OK if one loop iteration depends on the result of the
previous iteration. 

However, such formulations prohibit parallelizing
the loop. 

Sometimes we can find a parallelizable formulation of the loop,
but sometimes we haven't found one yet.

\end{frame}




\begin{frame}
\frametitle{Two Parts}

Finally, code often contains a sequential part and a parallelizable
part.  

If the sequential part dominates, then
executing the parallelizable part on infinite CPUs isn't going to speed up the task as a whole. 

\begin{center}
	\includegraphics[width=0.2\textwidth]{images/speedlimit.png}
\end{center}

This is
known as Amdahl's Law, and we'll talk about this soon.


\end{frame}



\begin{frame}
\frametitle{To Avoid Complications...}

 It's already quite difficult to make sure that
sequential programs work right. 

Making sure that a parallel program
works right is even more difficult.

The key complication is that there is no longer a total ordering between
program events. 

Instead, you have a partial ordering:\\
\quad Some events $A$ are guaranteed to happen before other events $B$.\\
\quad Many events $X$  and $Y$ can occur in either the order $XY$ or $YX$.

\end{frame}



\begin{frame}
\frametitle{Races}

 A \alert{data race} occurs when two threads or processes both attempt to
simultaneously access the same data.

At least one of the accesses is a write. 

This can lead to nonsensical intermediate states becoming
visible.

Avoiding data races requires
coordination between the participants to ensure that intermediate
states never become visible (typically using locks). 


\end{frame}



\begin{frame}
\frametitle{Deadlock}

 \alert{Deadlock}
occurs when none of the threads or processes can make progress.

There is a cycle in the resource requests. 

To avoid a deadlock, the programmer needs to enforce an ordering in the locks.


\end{frame}



\begin{frame}
\frametitle{But Will it Scale?}

It gets worse. Performance is great, but it's not the only thing. 

We also care about \alert{scalability}: the trend of performance with increasing load. 

A program generally has a designed load (e.g., $x$ transactions per hour). 

A properly designed program will be able to meet this intended load. 

If the performance deteriorates rapidly with increasing load (that is, the number of operations to do), we say it is \alert{not scalable}.

\end{frame}



\begin{frame}
\frametitle{Scaling Up}

If we have a good program design it can be fixed. 

If we have a bad program design: ``rearranging deck chairs on the Titanic''.

Even the most scalable systems have their limits, of course, and while higher is better, nothing is infinite.

\end{frame}

\begin{frame}
\frametitle{Laws of Performant Software}

Suppose you want to write fast programs...\\
\quad And you like checklists and handy rules. 

If so, you are in luck, because there is Crista's Five Laws of Performant Software!

\begin{center}
	\includegraphics[width=0.5\textwidth]{images/thelaw.png}
\end{center}

\end{frame}



\begin{frame}
\frametitle{Rule 1: Knowledge is Power}

\textbf{1. Programming language $\ll$ Programmers' awareness of performance.}

There is no programming language that is magic, whether good or evil. 

All the major programming languages allow you to write programs that perform well or badly. 

High level languages give you lots of options...\\
\quad Do I use an array? A vector? A list? 

What do they do behind the scenes?\\
\quad Is there a better way?

\end{frame}



\begin{frame}
\frametitle{Rule 1 Asterisk}

\begin{center}
	\includegraphics[width=0.2\textwidth]{images/lambda.png}
\end{center}

Some languages lend themselves better to parallelization than others. 

A language may force a certain way of thinking, based on its rules (e.g., functional programming languages). 

But there is no reason why the way of thinking can't be applied in another language.


\end{frame}



\begin{frame}
\frametitle{Rule 2: Nonlinear Output}

\textbf{2. $d(f^{\tau}(x), f^{\tau}(y)) > e^{\alpha\tau} d(x, y)$ or small details matter.}

If two versions of the code are $x$ and $y$, the difference between the performance outcomes $f(x), f(y)$ is much larger than the difference between the code.

Did you fix a memory leak? The addition of one \texttt{free()} call is a single line code change but can, in the long run, have a dramatic impact on performance. 

Don't overlook the small stuff!

\end{frame}




\begin{frame}
\frametitle{Rule 3: Establish Boundaries}

\textbf{3. corr(performance degradation, unbounded resource usage) > 0.9.}

There is a very high correlation between performance degradation and unbounded use of resources. 

Often times we focus on functionality: the software must have the following 847~251 features! 

But if you want a program that scales you need to think in terms of operation, not functionality.

\end{frame}



\begin{frame}
\frametitle{Rule 3: Establish Boundaries}

Resources need to be limited. 

If there aren't hard limits, eventually a resource will be exhausted. 

If the program starts threads, use a thread pool and the thread pool should have a fixed size. 

Is there a cache? It needs a maximum size. 

If you need to read input, don't use a function that reads an entire line (of arbitrary length). 

\end{frame}



\begin{frame}
\frametitle{Rule 3: Establish Boundaries}

Furthermore your program needs design effort given to what happens when resources are exceeded. 

So you decide to set a request queue size; once that queue is full, further requests are rejected in some well-understood manner. 


\end{frame}



\begin{frame}
\frametitle{Rule 4: Do Experiments}

\begin{center}
	\includegraphics[width=0.3\textwidth]{images/ndg.jpg}
\end{center}

If you want your code to be faster you have to know why it is slow. 

It's okay not to know the answers, but not knowing how to find out is a problem. 

Don't guess; measure.

\end{frame}



\begin{frame}
\frametitle{Rule 5: You Can't Outrun the Fork}

\textbf{5. N*bad != good.}

No amount of nodes, cores, memory, etc, will save you from poorly-written code.

Throwing more hardware at the problem is expensive and ineffective in the long term. 

Bad code is still bad no matter how much hardware it runs on. 

\end{frame}

\begin{frame}
\frametitle{Rust}

\begin{center}
	\includegraphics[width=0.5\textwidth]{images/Rust.png}
\end{center}

We've used C and \CPP~ a lot in your program.

They lack some of the niceties you might like to have.

\end{frame}


\begin{frame}
\frametitle{C is Hard}

The nature of the languages make it hard, or even impossible, to write code that is fast, correct, and secure. 

\begin{center}
	\includegraphics[width=0.5\textwidth]{images/mando.jpg}
\end{center}

\end {frame}

\begin{frame}
\frametitle{C is Hard}
The focus of this course hasn't been on security. 

But in many cases, writing insecure fast code isn't the right thing. 

Is it even possible to write secure C and \CPP?

\end{frame}


\begin{frame}
\frametitle{Experience!}

Robert O'Callahan\footnote{Holds a PhD in CS from Carnegie Mellon University; was Distinguished Engineer at Mozilla for 10 years; now at Google; etc.)}: ``I cannot consistently write safe C/\CPP~ code.'' (17 July 2017)

What about use of better tools and best practices?

\end{frame}


\begin{frame}
\frametitle{Use Tools}

\begin{center}
	\includegraphics[width=0.4\textwidth]{images/chrome-vuln.png}
\end{center}

March 2019: disclosure of Chrome use-after-free vulnerability. 

0-day attacks observed in the wild.

Google implements best practices, and has all the tools and developers that
money can buy!

\end{frame}

\begin{frame}
\frametitle{Make Mistakes Impossible}

\begin{center}
	\includegraphics[width=0.4\textwidth]{images/plam-trying-harder.jpg}
        {\scriptsize pic: Elliott O'Brien}
\end{center}
``Try Harder''?

 Expecting people to be perfect and make no mistakes is unrealistic.
 
  What we want is to make mistakes impossible.


\end{frame}


\begin{frame}
\frametitle{Wait, we know this...}

A lot of the problems we frequently encounter are the kind that can be found by Valgrind, such as memory errors or race conditions.

Other tools like code reviews and Coverity (static analysis defect-finding tool) exist.

Can we solve it all at once?

At compile time?

\end{frame}



\begin{frame}
\frametitle{Rust}

A design goal of this language is to avoid issues with memory allocation and concurrency. 


It does so by checking things at compile time that most languages don't check at all, and if so, only at runtime.

\end{frame}



%% \begin{frame}
%% \frametitle{Recap: The Roadmap}
%% Baseline understanding of hardware: architecture, caches, branch prediction.

%% Parallelize your program: threads, locking (but do it well!).

%% Speculation, single thread performance, OpenCL.

%% Profiling tools to find where and what to focus on next. 

%% Use multiple machines: such as with MPI, queueing theory.

%% \end{frame}

\begin{frame}[fragile]
\frametitle{Recap: The Roadmap}
\begin{tikzpicture}[remember picture, overlay]
\node[draw] (hw) at ([xshift=-3cm, yshift=-2.5cm]current page.north) { \begin{tabular}{c}1. Learn some Rust\end{tabular}};

\node[draw] (hw) at ([xshift=0.5cm, yshift=-3cm]current page.north) { \begin{tabular}{c}2. Grok the HW \\ (ECE 222++)\end{tabular}};

\node[draw] (par) at ([xshift=-1.5cm, yshift=-1.5cm]hw) { \begin{tabular}{c} 3. Parallelize code well.\\ (threads, locking) \end{tabular}};

\node[draw] (misc) at ([xshift=.5cm, yshift=-1.5cm]par) { \begin{tabular}{c} 4. Speculation, 1-thread perf, CUDA \end{tabular}};

\node[draw] (prof) at ([xshift=3.5cm, yshift=-1.5cm]hw) { \begin{tabular}{c} 5. Profiling \\ (what's slow?) \end{tabular}};

\node[draw] (dist) at ([xshift=1.5cm, yshift=-1.5cm]misc) { \begin{tabular}{c} 6. Use many machines \\ (Cloud Computing, queueing theory) \end{tabular}};


%% Baseline understanding of hardware: architecture, caches, branch prediction.

%% Parallelize your program: threads, locking (but do it well!).

%% Speculation, single thread performance, OpenCL.

%% Profiling tools to find where and what to focus on next. 

%% Use multiple machines: such as with MPI, queueing theory.
\end{tikzpicture}

\end{frame}


\end{document}

